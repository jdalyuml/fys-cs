\documentclass[10pt]{article}
\usepackage{amsmath, amssymb, graphics, verbatim, alltt, listings,
  framed, enumitem, hyperref}
\usepackage{soul}
\newcommand{\courseno}{COMP.1110}
\newcommand{\coursename}{$1^{st}$ Year Seminar}
\newcommand{\fullcoursename}{First Year Seminar for Computer Science Students}
\newcommand{\semester}{Fall 2020}
\newcommand{\dow}{M}
% \courseno \coursename \semester
\newcommand{\mathsym}[1]{{}}
\newcommand{\unicode}[1]{{}}
\pagestyle{myheadings}
\markright{
  \courseno{} \coursename \\ UMASS - Lowell \\ \semester
}
\addtolength{\oddsidemargin}{-0.875in}
\addtolength{\evensidemargin}{-0.875in}
\addtolength{\textwidth}{1.75in}
\addtolength{\topmargin}{-0.875in}
\addtolength{\textheight}{1.75in}
\begin{document}
  \begin{center}\Huge

  \end{center}
  \begin{enumerate}
    \item {\bf Course Number and Title:} \courseno{} \fullcoursename{}
    \item {\bf Instructor's Name:} Dr. James Daly.\quad{}email:
      \url{mailto:james_daly@cs.uml.edu} \\\\
      Please inculde the course number and section in the subject of
      your email.
    \item {\bf Instructor's Office:} DAN-347
    \item {\bf Office Hours:} 
      \vspace*{5mm}
% \begin{center}
  % \begin{tabular}{|lcc|}\hline
   % Monday    & 10:00 - 10:50, 12:00 - 12:50  &\\\hline
   % Tuesday   & 7:45  - 10:45  &\\\hline
   % Wednesday & 10:00 - 10:50, 12:00 - 12:50  &\\\hline
   % Thursday  & 7:45  - 10:45  &\\\hline
   % Friday    & 10:00 - 10:50, 12:00 - 12:50  &\\\hline
  % \end{tabular}
% \end{center}
% \vspace*{5mm}
% %Please send email to make an appointment.  
% These times are reserved for you, but I may have to attend a
% meeting from time to time.  
% If you are having difficulty with the material, I would
% \emph{strongly} encourage you to make use of office hours for
% additional help. 
By appointment over Zoom.  Please provide a prefered time and an alternate time in your email.

%    \item {\bf Rationale:} 
%      \input{Syllabus-parts/Rationale}
    \item {\bf Course Description:} 
        This is an introduction to both the field of Computer Science and
   the department.  The goal is to aid in the transition from high
   school to the University.  An overview of various opportunities
   will be given.
   \item {\bf Course Co-requisites:}
  None.
    \item {\bf Course Prerequisites:}
  None.
    \item {\bf Course Category:}
  Required.

%    \item {\bf Course Competencies:}  
%      Upon successful completion of this course, the student should learn how to:
\begin{itemize}
\item[$\bullet$] Read and review peer-reviewed articles from various
  sources (IEEE, ACM, various conferences).
\item[$\bullet$] Produce a project proposal
\item[$\bullet$] Provide timely status reports
\item[$\bullet$] Undertake a significant project and produce a
  write-up with appropriate parts (abstract, related work, problem
  description, references). 
\item[$\bullet$] Give a professional technical presentation to
  accompany the project paper
\end{itemize}


%    \item {\bf Essential Questions:} 
%       \input{Syllabus-parts/EssentialQuest}
      
    \item {\bf Student Outcomes:}  
      \begin{enumerate}[label=\arabic*.]
      \item Understand what ``Computer Science'' is.
        
      \item Become familiar with University and departmental policies. 
        
      \item Understand the basic programming and debugging techniques.
        
      \item  Develop a long range plan (5+ years) for themselves.

      \item  Explain what they do to people that are not in the field.

      \end{enumerate}

    \item {\bf Date, Time and location:} 
      
\begin{tabular}{|lcc|}\hline
Sec 203   & Wednesdays & 12:00-12:50pm \\\hline
Sec 205   & Wednesdays & 2:00-2:50pm \\\hline
Sec 206   & Tuesdays & 9:30-10:20am \\\hline
\end{tabular}

Class meets on Blackboard Collaborate.
    \item {\bf Required Textbook:} 
      {\bf None}




    \item {\bf Recommended Books:}
      Frederick P. Brooks {\em The Mythical Man-Month: Essays on Software
                  Engineering, Anniversary Edition (2nd Edition)},
  {Addison-Wesley Professional}, 1995,
  ISBN:  0201835959\\\\




         \item {\bf Evaluation Methodology:}\\
       Grades will be given on a contract basis.
To receive an A, you must attend at least 11/13 classes, do 5/6 of the homework assignments, and \st{attend at least 4 CS Colloquia, department talks, or UML-ACM presentations}.  
Homework is graded on a credit/no-credit basis.
Do not wait until the end of the semester to attend events as there may not be sufficient events for you to attend.

Students who meet all of the requirements for one rank and some (but not all) of the requirements for the next rank will have their grade raised by one step for each criteria (e.g. from a B to a B+).

\begin{center}\begin{tabular}{|l|c|c|c|}\hline
    Letter grade & Attendance & Homework & Colloquia\\
    \hline
    A & 11/13 & 5/6 & \st{4} 0  \\
    B & 9/13 & 4/6 & \st{3} 0 \\
    C & 7/13 & 3/6 & \st{2} 0  \\
    D & 6/13 & 2/6 & \st{1} 0 \\
    \hline
\end{tabular}\end{center}

For the \emph{Fall 2020} semester, the Colloquia requirement is waived.
     \item {\bf Due Dates:} 
       Homework is due on Blackboard at the times indicated (usually the beginning of class).  Late assignments will {\em not} be accepted without
prior approval.  Students must consult with the instructor at least
three days prior to the scheduled due date to make alternative
arrangements.  The instructor is under no obligation to grant any such
request.

     \item {\bf Exams:} 
       {\bf None}




    \item {\bf Course Expectations:}
       \begin{enumerate}[label=\alph*.]
         \item {\bf Attendance:}
           The classroom is the heart of the educational experience
           because it provides, uniquely, a formal 
           setting for the important exchanges among faculty and
           students. Regular and punctual attendance at all classes,
           essential for maximum academic achievement, is a major
           responsibility of the students. Failure to
           attend and contribute to the classroom environment
           significantly and demonstrably reduces the quality of the
           educational experience for everyone in the classroom. As a
           result, absences almost always impact the quality of
           performance.  Students are expected to arrive on-time and
           with all required material for the taking of notes (pencil
           and notebook preferred).
%           Each unexcused absence  will result in the student's
%           final grade being reduced by $1\%$. Three unexcused
%           absences are permitted during the semester.  Additional
%           unexcused absences may result in a failing grade and/or
%           removal from the course. Additionally, each
%           tardiness (for any reason) will count as one half of
%           an absence. An unexcused absence occurring on a day when
%           an exam or quiz is given will result in a grade of zero. In
%           the case of an excused absence, it is the student's
%           responsibility to work with the instructor to schedule a
%           makeup exam.  It is always in the student's best interest
%           not to miss the regularly scheduled in-class exam.\\\\ 
%           In the event of prolonged illness, accident, or similar
%           emergency, it is the responsibility of the student to
%           notify the instructor. Students must remember
%           that it is always their 
%           responsibility to make up the work they may have missed
%           during an absence from class. Students are directed to
%           confer with their instructor when their absences jeopardize
%           satisfactory progress.\\\\ 
         \item {\bf Academic Honesty Policy:}
           Integrity and honesty are extremely important qualities of
           any academic or professional person.  That being said,
           science is a collaborative effort. I encourage students to
           discuss course material outside of class and form study
           groups for exams and quizzes. {\bf However, you must do the
           homework by yourself. If you need help, I will be happy to
           help you!}. 
           %You may also seek assistance from the course  Teaching Assistants.
           \\\\
           {\bf The use of material from previous classes, solution
             manuals, material from the Internet or other sources (e.g.,
             parents, siblings, friends, etc.) that directly bears on
             the answer is strictly prohibited.}\\\\ 
           Examples of academic
           misconduct include copying from another student's exam,
           using unauthorized information from notes, calculators,
           phones, or computers during exams, copying all or part of
           another student's assignments such as code or diagrams and
           providing one's work to other students for their use.
           These examples are not exhaustive. \\\\
           The Computer Science Department considers any instance of
           academic misconduct to be a serious offense.  The penalty
           for an instance of academic misconduct may include, among
           other things, a zero for the assignment(s) in question, a
           failing grade for the course in question, or expulsion from
           the University.\\\\  
           Allowing another student to examine and/or copy your work
           constitutes an instance of academic misconduct both by you
           and by the other student. Thus academic misconduct includes
           not only taking others' work and submitting it as your own,
           but also allowing others to use your work or submit your
           work as their own. Students allowing others to copy their
           work will be charged with academic misconduct. They are
           subject to the same penalties listed.  Each student is
           responsible for safeguarding his/her own work.  The use of
           a personal flash drive is recommenced if you will be using
           a shared computer.\\\\  
           Examining and/or copying the work of another without their
           permission constitutes, in itself, a separate instance of
           academic misconduct. Thus it is not permissible to examine
           the work of others which is found either in printed form,
           as in a lab, or in electronic form, as on a public hard
           drive. Students who take the work of others in this manner
           will be charged with academic misconduct. They are subject
           to the same penalties listed above.\\\\
         \item {\bf Lecture Notes and Handouts:}
           Only a part of the lecture notes will be available in
           traditional digital-file formats (*.doc, *.ppt, and *.pdf)
           on the BlackBoard site related to our course:
           \url{https://uml.edu/myuml}.  Some handouts 
           will be distributed in class and some notes will be written
           on the classroom board.  Please, take notes every class and
           share the notes if someone needs them.  \\\\
         \item {\bf University Email System:}
           The University of Massachusetts - Lowell has established a University
           electronic mail (``email'') system as a means of the University
           sending official information to enrolled students, and for
           students to send communication to their instructors and
           University personnel. All registered students will be
           assigned a University email account/address to be used as the
           only email address for all email communication: 1) sent to
           the students from their instructors and from all University
           personnel; and, 2) sent by the students to their
           instructors and to all University personnel. \\\\
           In addition:\\
           \begin{enumerate}
           \item[$\bullet$] Students should check their University email
             account regularly to ensure they are staying current with
             all official communications. Official communication
             includes, but is not limited to, policy announcements,
             registration and billing information, schedule changes,
             emergency notifications and other critical and time
             sensitive information.  
           \item[$\bullet$] Students should also check their University
             email account to be sure that they are current with all
             email communication from their faculty.  
           \item[$\bullet$] The student email account/address should be
             the only e-mail address students use to send email to
             faculty and University personnel so that student email is
             recognized and opened. 
           \item[$\bullet$] Accounts are for individual use
             only, and are not transferable or to be used by any other
             individual.  
           \end{enumerate}

         \item {\bf Electronic Devices:}
           Regarding the use of electronic devices (such as cell
           phones, PDAs, pagers, MP3/iPods, laptops, etc.), students
           may not use these or other electronic devices during class
           unless permitted by the course instructor. If use of these
           devices is permitted by the instructor, they are to be used
           for appropriate class activities only. Augmentative communication
           devices are excluded from this policy (please refer to the
           Student Handbook policy on disability regarding these). If
           an emergency situation requires students to leave a cell
           phone on, they should inform the course instructor at the
           beginning of the class and leave the phone in a
           non-intrusive mode so as not to disrupt the class.
           In this class, {\bf NO electronic devices are permitted}
           during quizzes or exams.  \\\\
         \item {\bf Plagiarism Policy: }
           Plagiarism is a serious violation of a student's academic
           integrity and the trust between a student and his or her
           teachers. Plagiarism is the act of a person presenting
           another person's work as if it were his or her own
           original work. Such acts of plagiarism include, but are
           not limited to: 
           \begin{enumerate}
           \item[1.] A student submitting as his or her own work an
             entire essay or other assignment written by another
             person. 
           \item[2.] A student taking word for word a section or
             sections of another person's work without proper
             acknowledgment of the source and that the material is
             quoted. 
           \item[3.] A student using statistics or other such facts
             or insights as if these were the result of the
             student's efforts and thus lacking proper
             acknowledgment of the original source. 
           \item[4.] The paraphrasing of another person's unique
             work with no acknowledgment of the original source. 
           \item[5.] Copying another student's work on a quiz or
             test. 
           \end{enumerate}

           When a student is found to have plagiarized an academic
           assignment, it will be up to each instructor to determine
           the penalty. Depending on the severity of the incident,
           this could range from a warning to a loss of credit for
           the assignment. In all cases of plagiarism, the student's
           program coordinator will automatically be notified and
           the incident will be documented. If any further incidents
           of plagiarism are reported to the student's program
           coordinator, additional sanctions will be imposed. These
           may include notification of the Dean; loss of credit for
           the course; 
           suspension or dismissal from a department program;
           academic probation; and/or expulsion from the
           University. \\\\
           Additional information regarding University Policies is
           available:
           \begin{enumerate}
             \item Academic Policies:\\\\
               \url{https://www.uml.edu/Catalog/Undergraduate/Policies/Academic-Policies/Academic-Policies.aspx} \\\\
               \item Kennedy College of Sciences Policies:\\\\
                 \url{https://www.uml.edu/Catalog/Undergraduate/Sciences/policy/default.aspx}  \\\\
               \item Computer Science Department Policies:\\\\
                 \url{https://www.uml.edu/Catalog/Undergraduate/Sciences/policy/Continuance-appeal-dismissal.aspx}\\\\
                 \item Advising Resources:\\\\
                   \url{https://www.uml.edu/Academics/Provost-office/Faculty-success/Professional-Engagement/advising-resources.aspx}  \\\\

           \end{enumerate}
       \end{enumerate}
     \item {\bf Schedule} (subject to adjustment):\\
       \begin{center}
\begin{tabular}{|l|l|}\hline
Week & Topic \\ \hline
1 & Introduction, Blackboard, Syllabi, Academic Calendar\\ \hline
2 & Collaborations vs Plagiarism in CS, Git and GitHub\\ \hline
3 & Centers for Learning / Academic Advising\\ \hline
4 & Conversations about foundational works in CS\\ \hline
5 & UML CS student clubs\\ \hline
6 & Centers for Learning\\ \hline
7 & CS Research Faculty\\ \hline
8 & Career Trajectories\\ \hline
9 & DifferenceMaker\\ \hline
10 & Co-op program \& Career Services\\ \hline
11 & Women and minorities in CS, representation, equality, and allyship\\ \hline
12 & Linux intro, the shell, filesystem, and scripting\\ \hline
13 & Getting technical help, preparing for the end of the semester\\ \hline
\end{tabular}
\end{center}

% \begin{center}\begin{tabular}{|l|l|l|l|}\hline
  % %           & Date  & Events \\
  % Week & Topic                          & Readings      & Events \\\hline
  % 1 & Introduction, BlackBoard          &               &  \\
    % & Syllabi, Academic Calendar        &               &  \\
    % & \hspace*{6cm}                     &               &  \\\hline
  % %
  % 2 & Sis, Time Management, punctuality &               &  \\
    % & We're not in \st{Kansas} high school anymore  &               &  \\
    % &                                   &               &  \\\hline
 % %
  % 3 & GPA Requirements \& Management    &               &  \\
    % & Advising Season, Honors, Math     &               &  \\
    % & minor, Data Science Option        &               &  \\\hline
  % %
  % 4 & Study Skills                      &               &  \\ 
    % & Test Taking Strategies            &               &  \\
    % & Testing Center                    &               &  \\\hline
  % % 
  % 5 & Linux \& Unix Basics              &               &  \\ 
    % & Emacs vs. Vi                      &               &  \\
    % &                                   &               &  \\\hline
  % %
  % 6 & Writing Code \& Debugging,        &               &  \\
    % & Tools, Version Control            &               &  \\
    % & Academic Honesty                  &               &  \\\hline
  % %
  % 7 & ACM Student Chapter               &               &  \\ %James Daly
    % &                                   &               &  \\
    % &                                   &               &  \\\hline
  % %
  % 8 & Career Services \&                &               &  \\
    % & Internships                       &               &  \\ %Lada Lau
    % &                                   &               &  \\\hline
  % %
  % 9 & Difference Maker                  &               &  \\ %Tom WIlkes
    % &                                   &               &  \\
    % &                                   &               &  \\\hline
  % %
  % 10 & CS Research Faculty              &               &  \\
    % &                                   &               &  \\
    % &                                   &               &  \\\hline
  % %
  % 11 & Office Hours, ``Hey Prof'',      &               &  \\
    % & Professional Interactions         &               &  \\
    % &                                   &               &  \\\hline
  % %
  % 12 & School, Home, Work Balance       &               &  \\
    % &                                   &               &  \\
    % &                                   &               &  \\\hline
   % %
  % 13 & Are we having fun yet?           &               &  \\
    % & Finding technical help: on-line   &               &  \\
    % & resources \& documentation        &               &  \\\hline
  % %
% % 14 &                                  &               &  \\\hline
  % %
  % %
% % \multicolumn{4}{|c|}{Final Exams} \\\hline
  % %
% \end{tabular}\end{center}

% % 

      % \begin{center}\begin{tabular}{|r|l|l|}\hline
Week & Days & Dates\\\hline
  1  & M & 9/9 \\\hline
  2  & M & 9/16\\\hline
  3  & M & 9/23\\\hline
  4  & M & 9/30\\\hline
  5  & M & 10/7\\\hline
  6  & T & 10/15\\\hline
  7  & M & 10/21\\\hline
  8  & M & 10/28\\\hline
  9  & M & 11/4\\\hline
 10  & M & 11/18\\\hline
 11  & M & 11/25\\\hline
 12  & M & 12/2\\\hline
 13  & M & 12/9\\\hline
% 14  & M \\\hline
% 15  & M \\\hline
     & Final Exams & 12/14--12/21\\\hline

\end{tabular}\end{center}


  \end{enumerate}
\end{document}
