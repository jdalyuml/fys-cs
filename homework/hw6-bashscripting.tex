\documentclass[12pt]{article}

\usepackage{datetime}

\usepackage[T1]{fontenc}
\usepackage{amsmath, amssymb, amsthm}
\usepackage{enumerate}
\usepackage{verbatim}
\usepackage{listings}
\usepackage{fullpage}
\usepackage{graphicx}
\usepackage[dvipsnames]{xcolor}
\usepackage{hyperref}
\usepackage{xspace}

\newcommand{\hwno}{6}
\newcommand{\duemonth}{12}
\newcommand{\duedate}{9}

\newcommand{\code}[1]{\texttt{#1}}

\newcommand{\filename}{\code{rjcount.bash}\xspace}

\title{Homework \hwno: Bash Scripting}
\date{Due: \dayofweekname{\duedate}{\duemonth}{\year}, \monthname[\duemonth] \duedate, 5pm}

\begin{document}
\maketitle

% GENERAL TODO FOR NEXT TIME
% Put all commands on their own line

In this homework, you will write a simple script to determine how many times that Romeo and Juliet are mentioned get in \emph{Romeo and Juliet} by William Shakespeare.

By now, you should have been exposed to Linux in your Computing I course.  If you haven't, then you should complete the Linux tutorial at \url{http://www.cs.uml.edu/~dlipman/linux-tutorial/} before getting started.

% Connect to cs.uml.edu using Putty and FileZilla.
\section*{Connecting to cs.uml.edu}
You will need to use a Bash terminal to test your work for this assignment.
If you used the \href{https://git-scm.com/}{Git} utility on Windows, it comes with a Bash shell.
On macOS and Linux, the default terminal shell is Bash.

Alternately, you can use the department server.
You can access it using an SSH utilty such as \href{https://www.chiark.greenend.org.uk/~sgtatham/putty/latest.html}{PuTTY}.
You will connect to cs.uml.edu and sign in using your CS username and password (not your UML email), which you should have gotten for Computing I Lab.
PuTTY includes a command-line file-transfer protocol (FTP) utility, but you would probably prefer to use a graphical application.
I use \href{https://filezilla-project.org/}{FileZilla} for this purpose.

% TO CHANGE FOR NEXT TIME
% Difference between local shell and connecting to a remote server
% How to connect to a remote machine
% Model for where files live
% Images for doing stuff


\section*{Scripting Basics}
As mentioned in the Linux Tutorial, Bash (Bourne-Again SHell) is a common command language interpreter on Linux environments.
For tasks that they expect to do often, people will often create shell scripts (\code{.sh} or \code{.bash} files) that they can use to repeat the task.
These scripts will contain a list of commands that will be run in sequence.

Create a file named \filename (Blackboard does not accept \code{.sh} files).  Save the \code{romeo.txt} file in the same directory.

\section*{The Whole Shebang}

Script files begin with a \emph{shebang} (hash-bang) that instructs which interpreter it was designed to run with (the file extension only hints at how it will be run rather than dictating it).
This will be in the form \code{\#!<interpreter> [optional-args]}.
Since we are writing a Bash script, we use \code{\#!/bin/bash}.
Put this as the \textbf{\textcolor{red}{first}} line of your script.

In the Bourne shell language, lines that begin with \code{\#} are comments.
Thus, the shebang is ignored by the interpreter itself when executing your script.
You can also use \# to add comments or descriptions to your script.
However, the command environment only looks at the first line when determining which interpreter to use, so the shebang must be on the first line.


You will also see \code{/bin/sh} used instead of \code{/bin/bash}.
However, there are many other shells (such as the Korn shell (ksh), Almquish shell (ash), Z shell (zsh) and C shell (csh)).
Thus, \code{/bin/sh} may be configured to point to one of these other shells instead, which would cause problems if you use Bash-only features.
Thus, it is better to explicitly list which interpreter you wish to use.

\section*{Execution test}
Add the line \code{echo Hello world} to your script, then run it by typing \code{./\filename} into the terminal 
(the initial \code{./} tells it to run a script in the current directory rather than looking where programs are usually stored).
If you get a permission denied error, you can use \code{chmod +x \filename} to add eXecutable permissions to the file.
Once you've verified that you can run the script, you can remove the \code{echo} command.

\section*{Simple Commands}
You can add simple commands, like \href{https://en.wikipedia.org/wiki/Echo_(command)}{\code{echo}} or \href{https://en.wikipedia.org/wiki/Wc_(Unix)}{\code{wc}} to do tasks like output a string or count the lines of a file.
Use the \code{wc} command to output the number of lines in the \code{romeo.txt} file.
You will need to use the \code{-l} flag to do this.
During this exercise, we will add to this command to output the number of lines that either Romeo or Juliet get.

Remember that if you forgot how to use a particular command, you can use the \href{https://en.wikipedia.org/wiki/Man_page}{\code{man}} on the \emph{terminal} (not the script) to look up the manual page for a particular command.
If you forgot which command does something, you can try to use the \href{https://en.wikipedia.org/wiki/Apropos_(Unix)}{\code{apropos}} (from the French ``\`{a} propos'' meaning ``about'') command to search for an appropriate command.

\section*{Piping}
It is often handy to take the results of one program and use them as input for another program.
You can use the pipe operator (\code{|}) to pass the output from one program to another.
For example, \code{ls | wc -l} will count how many files are in the current directory.

Other times, we might want to read to or write from a file.
You can use the \code{<} operator to read from a file and the \code{>} operator to write to a file.
For example, \code{ls > files.txt} will write the names of all of the files into \code{files.txt} and \code{wc -l files.txt} will count how many lines end up in that file.

You can use piping to count the number of lines in \code{romeo.txt}.

\section*{Grep}

\href{https://en.wikipedia.org/wiki/Grep}{\code{grep}} is a powerful utility that allows you to find lines in a file that match a particular \href{https://en.wikipedia.org/wiki/Regular_expression}{regular expression}.
We're just going to look for particular words rather than fitting patterns, so we won't be using its full power.
The basic usage is \code{grep <pattern> <filename>}.
For example, we could use \code{grep Smith members.txt} to find all of the people named Smith in a list of club members.
We could also output redirection to get input from another command, so \code{ls | grep hw} would list all homework in the current directory.

The default is case-sensitive.
You can use the \code{-i} flag to do case-insensitive matching.

You can now use \code{grep} to count the number of lines mentioning Romeo.

\section*{Variables}

Sometimes you need to store the result of a calculation in a variable.
We can set our own variables with \code{<varname>=<value>} but we \emph{use} them by putting a \code{\$} in front of the variable name.
For example, we can set a message with \code{msg=World} and print it with \code{echo Hello \$msg}.

There are some special variables.
\code{\$0} is the name of the script, \code{\$1} is its first argument, \code{\$2} is the second, and so on.
For example, if you ran \code{script.sh Hi 5} then \code{\$1} would evaluate to \code{Hi} and \code{\$2} would evaluate to \code{5}.

Use variables so that the user can enter the character to find and the file to search through.
For example, running \code{./\filename Romeo romeo.txt} should print 301 and \code{./\filename Juliet romeo.txt} should print 181.
If you have received 121 and 47, remember to do a case-insensitive search.

\section*{Submission}

Submit your completed \filename script on Blackboard.

\end{document}