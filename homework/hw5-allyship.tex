\documentclass[12pt]{article}

\usepackage{datetime}

\usepackage[T1]{fontenc}
\usepackage{amsmath, amssymb, amsthm}
\usepackage{enumerate}
\usepackage{verbatim}
\usepackage{listings}
\usepackage{fullpage}
\usepackage{graphicx}
\usepackage[dvipsnames]{xcolor}
\usepackage{hyperref}

\newcommand{\hwno}{5}
\newcommand{\duemonth}{11}
\newcommand{\duedate}{24}

\newcommand{\code}[1]{\texttt{#1}}

\title{Homework \hwno: Allyship}
\date{Due: \dayofweekname{\duedate}{\duemonth}{\year}, \monthname[\duemonth] \duedate, 5pm}

\begin{document}
\maketitle

For this assignment, you will read about diversity and allyship and then reflect on how to build an inclusive classroom.

Read \href{https://embodyedu.wordpress.com/2020/06/29/what-does-educational-allyship-look-like/}{\emph{What does being an ally look like at school?}} by Caitlin O'Connor, 
\href{https://blog.upperlinecode.com/supporting-girls-in-computer-science-bafba150989e}{\emph{Code Like a Girl: Supporting young women in the computer science classroom}} by Marieke Thomas, and 
\href{https://dearally.com/2019/08/20/14-how-do-i-talk-to-people-who-are-against-allyship-as-a-concept/}{\emph{How do I talk to people who are against allyship as a concept?}} by Valerie Aurora.

Afterwards, write a response in which you reflect what a diverse classroom means to you.  Be sure to include both what you can do to make others feel more included and what others can do to make you feel like you belong.

Submit your response on Blackboard.

\end{document}